%\begin{Pre_Settings}
\documentclass[UTF8]{ctexart}
% \usepackage{CJK}
\usepackage{fancyhdr}
\usepackage{extramarks}
\usepackage{amsmath}
\usepackage{amssymb}
\usepackage{latexsym}
\usepackage{amsthm}
\usepackage{amsfonts}
\usepackage{tikz}
\usepackage[plain]{algorithm}
\usepackage{algpseudocode}
\usepackage{geometry}
\usepackage{graphicx}


\usepackage{subfigure}
\usepackage{booktabs}
\usepackage{indentfirst}
\usepackage{tabularx}
\usepackage{array}
\usepackage{listings}
\usepackage{xcolor}

\usepackage{calc}


% (1) choose a font that is available as T1
% for example:
\usepackage{lmodern}
% (2) specify encoding
\usepackage[T1]{fontenc}
% (3) load symbol definitions
\usepackage{textcomp}
%\end{Pre_Settings}


%\begin{Code_Settings}
\lstset{ %
language=C++,                % the language of the code
basicstyle=\footnotesize,           % the size of the fonts that are used for the code
numbers=left,                   % where to put the line-numbers
numberstyle=\tiny\color{gray},  % the style that is used for the line-numbers
stepnumber=2,                   % the step between two line-numbers. If it's 1, each line 
% will be numbered
numbersep=5pt,                  % how far the line-numbers are from the code
backgroundcolor=\color{white},      % choose the background color. You must add \usepackage{color}
showspaces=false,               % show spaces adding particular underscores
showstringspaces=false,         % underline spaces within strings
showtabs=false,                 % show tabs within strings adding particular underscores
backgroundcolor=\color[RGB]{245,245,244},
framexleftmargin=0mm,
frame=none,       % if not set, the frame-color may be changed on line-breaks within not-black text (e.g. commens (green here))
tabsize=4,                      % sets default tabsize to 2 spaces
captionpos=b,                   % sets the caption-position to bottom
breaklines=true,                % sets automatic line breaking
breakatwhitespace=false,        % sets if automatic breaks should only happen at whitespace
title=\lstname,                 % show the filename of files included with \lstinputlisting;
% also try caption instead of title
keywordstyle=\color{blue},          % keyword style
commentstyle=\color[RGB]{0,96,96},       % comment style
stringstyle=\color{orange},         % string literal style
escapeinside={\%*}{*)},            % if you want to add LaTeX within your code
morekeywords={*,...}  
}
%\end{Code_Settings}


%\begin{Passage_Settings}
\CTEXsetup[format={\Large\bfseries}]{section}

% Basic Document Settings
\topmargin=-0.45in
\evensidemargin=0in
\oddsidemargin=0in
\textwidth=6.5in
\textheight=9.0in
\headsep=0.25in

\linespread{1.1}

\pagestyle{fancy}
\lhead{\hmwkSubject}
\chead{\hmwkSemester}
\rhead{\hmwkClassNum}
\lfoot{\lastxmark} 
\cfoot{\thepage}

\renewcommand\headrulewidth{0.4pt}
\renewcommand\footrulewidth{0.4pt}
\newcolumntype{Y}{>{\centering\arraybackslash}X} 


\setlength\parindent{0pt}

\title
{
    \vspace{2in}
    \textmd{\textbf{\hmwkClass:\ \hmwkTitle}}\\
    \normalsize\vspace{0.1in}\small{Due\ on\ \hmwkDueDate\ }\\
    \vspace{1in}
    }
%\end{Passage_Settings}


%\begin{init}
\newcommand{\hmwkSemester}{2019年夏季学期}
\newcommand{\hmwkClassNum}{智能家居终端}
\newcommand{\hmwkSubject}{《电子系统设计》}
\newcommand{\chinesedash}{\rule[.7ex]{\widthof{二字}}{0.5pt}}
%\end{init}
    

%\begin{main}
\begin{document}
    
%Titlepage
\begin{titlepage}
    \begin{center}
        \phantom{Start!}
        \vspace{2in}
        \center{\zihao{-1}\textbf{TARS智能家居终端}}
        \newline
        \newline
    \rightline{\zihao{-1}\textbf{\chinesedash 电子系统设计课程报告}}
    \setlength{\baselineskip}{40pt}
    \vspace{1.5cm}
    \zihao{-2}
    \vspace{1.5cm}
    \center{
        \begin{tabular}{cl}
        组号:&  666  \\\\
        姓名:& 张雨阳、张亦弛、孙玉东\\\\
        日期:&  2019.8.30 \\\\
        \end{tabular}
        }
    \end{center}
\end{titlepage}
\pagebreak
%end_Titlepage

\section{选题背景与意义}
    \hspace{1.5em}在这个信息时代,万物互联,织成了一张无形的大网。身处其中的我们,只要说几句话、敲几下键盘,就能轻松借助物联网完成想做的事。相形之下,我们的日常生活就显得有所欠缺。基于此,我们决定制作一款智能终端(智能家居机器人终端),把这样的科技感与便捷感带进日常生活。

    \hspace{1.5em}我们智能终端的灵感来源于星际穿越中的智能机器人TARS,它将能够完成家居环境中的电器系统管理,并提供远程遥控、语音控制的双重支持,为我们的生活带来极大的便利。

\section{系统架构设计与方案比选}
    \hspace{1.5em}如图所示,整个智能终端系统有输入、处理、输出三个层次。我们提供了“语音控制”以及“远程控制”两种输入方式;“移动”、“电器控制”、“用户反馈”三种输出形式。下面一一介绍。
    \subsection{输入}
    \subsubsection{语音控制}
    \hspace{1.5em}经过调研,我们起初准备使用LD3320语音识别模块进行识别,并搭配XFS5152语音合成模块以实现发声的功能。但由于两个模块显得冗余,加之成本过高、收货时间过晚等问题,最终选择了集成程度更高的NEWWAY语音识别模块。

    \hspace{1.5em}NEWWAY语音识别模块集成了语音识别、语音合成、播放功能,并提供了程序开发文档以及相应的IDE。利用这些硬件、软件资源,我们只需要做一些顶层开发即可实现智能终端需要的功能。

    \hspace{1.5em}于是我们设计了如下图所示的语音控制逻辑框架。此逻辑框架由五个模态组成,分别为“初始化模态”、“TARS模态”以及“睡眠模态”。
    \newline 

    \hspace{1.5em}智能终端开启后即进入\textbf{“初始化模态”}。在初始化模态中,终端能够识别“TARS”和“睡眠”两个关键词并进入相应模态。如果在初始化模态中停留超过15分钟没有进行任何动作,则自动进入睡眠模态。
    \newline
    
    \hspace{1.5em}进入\textbf{“TARS模态”}之后,智能终端可以识别一系列语音指令并向单片机发出消息(具体通信协议详见附件中“通信协议.pdf”):
    \begin{itemize}
        \item 行进指令
        \begin{enumerate}
            \item “前进”:向单片机发送“前进一步”的指令
            \item “后退”:向单片机发送“后退一步”的指令
        \end{enumerate}
        \item 空调控制指令:
        \begin{enumerate}
            \item “开空调”:向单片机发送“打开空调”的指令
            \item “关空调”:向单片机发送“关闭空调”的指令
            \item 温度控制指令:“空调X度”,则向单片机发送“将空调调节至X度”的指令
        \end{enumerate}
    \end{itemize}
    进入TARS模态如果超过15秒没有收到指令,则自动返回初始化模态。
    \newline

    \hspace{1.5em}进入\textbf{“睡眠模态”}后,智能终端停止接收上述指令。当识别到关键词“TARS”后,重新返回初始化模态。

    \subsubsection{远程控制}
    \hspace{1.5em}远程控制有红外、蓝牙和无线控制几种选择。其中红外与蓝牙都只能提供本地的通信,不能为“人与智能终端处于异地”的场景提供解决方法。所以我们最终选择了无线控制。

    \hspace{1.5em}为了实现无线控制,我们选择了Particle Photon开发板:通过电脑端向云端发送指令,云端与Photon开发板进行通信。Photon开发板接收到指令后向主板发出信号,最终实现控制智能终端的功能。

    \hspace{1.5em}为了从电脑端与Photon顺利通信,我们设计了自己的通信协议(详见附件“通讯协议.pdf”),每条消息由指令头和指令消息构成。通过解析指令头,获取消息的接受对象、消息条数;再据此解析指令消息,最终得到所需的指令。

    \subsection{处理}
    \hspace{1.5em}所有的消息都会通过串口最终发送到主板“Arduino Mega”上,由主板根据通信协议解码后转化为相应的输出。

    \subsection{输出}
    \subsubsection{移动}
    \hspace{1.5em}智能终端最基本的输出形式就是机器人的移动,移动的说明以前进为例。
    
    \hspace{1.5em}智能终端的一次前进分为七个步骤,完全由机身上的四个舵机带动。首先抬起两条侧腿,并向前转动。转动后,侧腿前方的配重将带动智能终端向前倾倒,中心落到处在前方的两条侧腿上。此后多次重复“抬起并向前转动中间腿”的动作,最终恢复到初始角度并达到前进效果。

    \subsubsection{电器控制}
    \hspace{1.5em}智能终端共设计了两种控制用电器的方式:直接控制用电器或通过安装可控插头控制用电器。限于时间,我们只实现了前者;对前者的说明以空调控制为例。

    \hspace{1.5em}我们通过终端顶部的红外发射管向空调发送相应的红外编码,即可实现对空调的控制。

    \subsubsection{反馈}
    \hspace{1.5em}智能终端会通过“语音”、“灯光”两种形式给用户反馈。语音通过NEWWAY的语音播放功能播放;灯光由终端正面的“无限灯”展示。其中无限灯由一面“半透半返镜”以及一块平面镜组成;将一个LED灯带置于其间,即可通过光线的不断反射,形成一条灯带的无数个像,达到“无限灯”的效果。
    
    \hspace{1.5em}开机(进入前述的初始化模态)后,终端会播放语音“欢迎使用TARS”的语音,同时无限灯将发白光(如图一所示)。在进入TARS模态后,无限灯将循环发出彩色光,表示正等待指令(如图二所示)。收到指令后,智能终端将执行相应指令并使无限灯发蓝光,表示收到指令(如图三所示)。进入睡眠模式后,终端会播放语音“TARS进入睡眠模式”,并使无限灯发白光,表示进入睡眠模式。

\section{机械结构设计}

\section{程序设计}
    \hspace{1.5em}对于核心处理器Arduino Mega,考虑到Arduino对于外设中断的支持较弱,所以主要采用阻塞式面向过程程序设计。在每个循环中先尝试读取由Photon和Newway发来的协议化串口消息,并对其中的数据进行解析、处理、执行;然后考虑到LED灯带有时需要渐变或连续改变亮度与颜色,所以在主循环体中也加入了LED灯带显示的相关代码。
    \begin{lstlisting}[language={C++}]
void loop() {
    Serial1Input();

    // led display
    /* some code here */
    FastLED.show();
    FastLED.delay(1000 / UPDATES_PER_SECOND);
}
    \end{lstlisting}
    \hspace{1.5em}具体说明对数据进行处理执行的过程,我们将每个输出模块的相关代码整理成单独的库文件,通过调用其中的函数接口来执行相关操作。其中包括了红外发送、行动前进指令、行动调试模式、LED灯带变色等所有在系统架构设计章节所提到的功能。
    \hspace{1.5em}对于网路端消息分发结点Particle Photon,其代码内容即是定义处理POST请求的相关函数接口,并将POST请求中所包含的信息重新整合后通过串口发送给核心处理器Arduino Mega。

\section{测试过程}
    \subsection{机械结构调试}
    \hspace{1.5em}在拼装机械结构的过程中,我们发现了一系列问题。对于侧腿左右摇晃的问题,我们增加了齿条挡板,控制了齿条的晃动,进而减小了侧腿晃动幅度;对于侧腿质量过大导致舵机与齿轮接口被磨损的问题,我们将舵机型号更换为“MG996R”(接口为金属材质),其接口材料的耐磨性有效的解决了材料磨损问题。此外,我们前期考虑不周,没有为终端正面的“无限灯”、背面的“电源开关”留出空间,最后重新切割相应模板解决了问题。

    \subsection{前进动作的调试}
    \hspace{1.5em}如上所述,一次前进主要为抬腿前倾、收腿归位两大部分,共分为七个小步骤。我们通过云端给系统中的Particle Photon发送debug指令,不断调节侧腿的上升距离、旋转角度参数,最终得到理想的结果。
    \hspace{1.5em}起初在相对光滑的地面上,机器人在收腿归位时会因动量守恒而无法前进,导致每一步前进距离过短。而在较粗糙的地面上则因阻力过大而几乎无法收腿。我们在前腿上增加了四块配重,使重心偏向前,增大了侧腿接触面的摩擦力,最终实现了正常步幅(每步约30cm)的前进。

    \subsection{红外控制的调试}
    \hspace{1.5em}我们先通过红外接收管,接收空调遥控器发出的红外编码。通过逐条比较,分析出编码与指令的关系(如温度控制位、模式控制位的位置)。最后根据这些规律得到我们需要的红外编码(打开空调、关闭空调、设置温度为27度等)。

    \hspace{1.5em}之后将得到的编码通过红外发射管向空调输出时,发现信号不起作用。通过深入分析,发现每条指令中除了数据位还包含了校验位。由于缺乏相应的分析知识,我们放弃了自行编码,最终选择手动记录每条指令对应的红外编码。将原始编码通过发射管发出后生效,测试成功。

    \subsection{电路修正}
    \hspace{1.5em}我们电路搭建完成后,进行电路测试却发现NEWWAY语音识别模块通过串口发送的消息无法被检测,只能收到Photon模块的串口消息。
    \hspace{1.5em}我们猜测可能是因为Photon内部的上下拉过强,导致NEWWAY串口信号被屏蔽。于是我们在Photon的TX端口串联了200R的限流电阻,并再次进行收发测试。此时发现两边数据都能被正常接收。修正成功。

\section{未来展望与应用前景}



\end{document}
%\end{main}